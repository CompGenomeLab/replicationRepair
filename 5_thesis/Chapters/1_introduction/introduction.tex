\setlength{\parindent}{0pt}
\chapter{\bf INTRODUCTION}

\section{UV-Induced Damages in Humans}

Ultraviolet (UV) light is the major cause of skin cancers in humans \citep{kiefer2007effects}. It is a portion of the electromagnetic (EM) spectrum which is emitted from the sun together with visible light and heat. Based on its wavelength, UV light divides into three subgroups: UVA (wavelength 315-400 nm), UVB (wavelength 280-315 nm), and UVC (wavelength 100-280 nm). While the less energetic UVA makes up the majority of UV light passing the atmosphere, all UVC and approximately 90\% of UVB is either blocked or absorbed by the ozone layer. Even in these conditions, we are not fully protected from the damaging effects of UV light. So that, UV irradiation accounts for approximately 30.000 DNA lesions’ formation per cell per hour. 

The most abundant UV lesions in cellular DNA are pyrimidine dimers \citep{kielbassa1997wavelength}, which are formed by the covalent bonds between the adjacent pyrimidines \citep{whitmore2001effect}. Different in their chemical structure, two types of pyrimidine dimers are exist; one is called cyclobutene pyrimidine dimers (CPDs), and the other is called pyrimidine (6-4) pyrimidone photoproducts [(6-4)PPs]. While both UVC and UVB can induce these dimer formations, UVA is only capable of induce CPDs. Nonetheless, UVA induction can convert already formed (6-4)PPs into their Dewar valence isomers. Moreover, UVA can induce oxidative DNA damages through photosensitized reactions \citep{hu2017cartography}. Thanks to the development of time resolved spectroscopy techniques in recent years, dynamics of pyrimidine dimer formation is well known. The formation and biological properties of UV lesions will be briefly discussed in the subsections below.   

\subsection{Cyclobutene Pyrimidine Dimers (CPDs)} 

CPDs are the most frequent pyrimidine dimers that are arising from the covalent linkages between the consecutive pyrimidines, and it is characterized by the four-member ring structure that are double bonded from the pyrimidine 5 and 6 \citep{whitmore2001effect}. In vivo, CPDs can be observed in four different configurations: cis-syn, cis-anti, trans-syn, or trans-anti. \citep{khattak1972photochemical} While it is generally observed in cis-syn form when the DNA is double-stranded \citep{wacker1964organic}, in denatured DNA and single-stranded regions, trans-syn configuration exists \citep{taylor1988synthesis}. Although it is rare, nonadjacent pyrimidines can also form CPDs in single-stranded regions \citep{nguyen1988ultraviolet}. Moreover, different configurations can affect the ability of repair enzymes to recognize these lesions and correct them, which cause mutability differences between the configurations \citep{friedberg2005dna}. 

Apart from the configuration of the lesions, their dipyrimidine doublets (TT, TC, CT, and CC) can contribute to CPD formation at different rates depending on the type of UV exposure or the nucleotide content of the DNA. According to the study of Douki and Cadet, under UVC and UVB exposure, double-stranded mammalian DNA produces TT, TC, CT, and CC CPDs in 100:50:25:10 ratios, respectively \citep{douki2001individual}. While TT CPDs accounts for more than half of the total CPDs after the exposure of UVC and UVB, for UVA exposure, this ratio rises to 90\% \citep{mouret2010uva}. On the other hand, TT CPDs are the abundant products of UV exposure for mammalian DNA, but the abundance might be greatly influenced by the GC percentage of the DNA. For example, in the bacterial DNA that possess a rich GC percentage, TT CPDs are the minor products of UV exposure \citep{patrick1977studies}.

\subsection{Pyrimidine (6-4) Pyrimidone Photoproducts [(6-4)PPs] and their Dewar Valence Isomers}

(6-4)PPs form with occurrence of a pyrimidone ring by the bonding between C6 position of the 5’-end base and C4 position of the 3’-end base. In fact, this structure forms indirectly following the UV exposure, after a cyclic reaction intermediate, which can be either an oxetane if thymine is the 3’-end base, or azetidine if cytosine is the 3’-end base. Because of its indirect formation, (6-4)PPs appear thousand times slower than CPDs \citep{schreier2007thymine}. 

Under UVC and UVB exposure, formation of (6-4)PPs is approximately five time less than that of CPDs \citep{douki2001individual}. Moreover, TT dipyrimidines that are the most abundant sites for CPDs are less frequent for (6-4)PPs. Instead, TC and CCs are the frequent sites for (6-4)PPs, while CT (6-4)PPs are uncommon. Another unique property of (6-4)PPs is its conversion into Dewar valence isomers with the photoisomerization process \citep{taylor1987dna}. Although UVB irradiation can trigger the process, with the combination of UVB and UVA exposure, the yield increases significantly.

\section{Nucleotide Excision Repair in Humans}

Throughout the generations, cells manage to evolve highly specialized repair mechanisms to cope with a variety of lesions that threaten the genome integrity and survival. Considering the diversity of these lesions, it would be unexpected to have only a single mechanism that can preserve the integrity of the genome. Hence, there are several repair mechanisms that cells utilize which are eminently conserved between species. Due to the removal of both strands, repair of a double strand break is demanding. There are two mechanisms that can be triggered by double strand breaks: homologous recombination, and non-homologous end-joining. Homologous recombination uses the sister-chromatid as a template to repair double strand breaks in an error-free manner. In addition, if sister-chromatid is not available for use, non-homologous end-joining directs the fusion of broken ends in an error-prone manner. Although being error-prone, non-homologous end-joining is the dominant mechanism for double strand break repair in mammals. Reasons of this dominancy are the distant proximity of chromatids to each other, and the DNA folding that makes the homologous sequence less reachable. In addition, imperfect matches by homologous recombination can lead to tragic outcomes such as creating repeated sequences \citep{li2018mismatch}.

On the other hand, when a damage occurs at a single strand, the opposite strand can be used as a template. In such cases, DNA excision repair mechanisms remove the lesion site and re-synthesize the gap using the template strand. While oxidation, deamination and alkylation damages are repaired by base excision repair \citep{klungland1999base}, mismatches that escape proofreading are identified and corrected by mismatch repair \citep{modrich1997strand}. And lastly, bulky adducts caused by UV irradiation, environmental mutagens, and chemotherapeutic agents are removed by nucleotide excision repair \citep{reardon2005nucleotide}. Nucleotide excision repair contains two sub pathways that differ from each other at the damage recognition step: Global Repair (GR) and Transcription-Coupled Repair (TCR). TCR is specialized in recognizing adducts in transcribed regions, while GR can recognize bulky adducts at any site. Subsections below will address the assembly and main properties of nucleotide excision repair in more detail.  

\subsection{Repair of UV-induced damages by Nucleotide Excision Repair}

Identified firstly at E. coli by two independent studies published in 1964 \citep{boyce1964release,setlow1964disappearance}, nucleotide excision repair can repair variety of bulky adducts from UV-induced pyrimidine dimers to chemotherapeutic agents such as cisplatin \citep{yimit2019differential}. Although repair mechanisms are highly conserved among the species, nucleotide excision repair in humans appeared to be surprisingly different from that of E. coli. While E. coli contains three proteins (UvrA, B, C) for the incision of damaged fragments, human nucleotide excision repair has sixteen proteins for the task. More interestingly, there is not an evolutionarily relevance between these human and E. coli proteins. In addition, the excised fragments are usually around 12 nucleotides long in E. coli. For humans, the length of these fragments are around 30 nucleotides \citep{sancar2016mechanisms}. Human nucleotide excision repair can be generally discussed in three steps: 1) damage recognition, 2) dual incision and excision of damaged fragments, and 3) re-synthesis and ligation.  

\subsubsection{Damage Recognition}

As it is mentioned earlier, GR and TCR have distinct damage recognition steps. GR scans the whole genome to detect helix distortions caused by bulky adducts, whereas TCR responds only to a stalled RNA polymerase II during transcription. 

In GR, three proteins (XPC, RAD23B, CETN2) work in coordination to recognize the lesion site \citep{sugasawa1998xeroderma}. XPC is the first protein to interact with the lesion by binding to the small single-stranded DNA (ssDNA) that is left unpaired due to the pyrimidine dimer formation at the opposite strand. The ability of XPC to bind unpaired ssDNA enables GR to detect a variety of lesions, since the unpaired ssDNA is a common characteristic of bulky adducts. After XPC binding, RAD23B and CETN2 interact with and stabilize XPC. However, helix distortions must be apparent to XPC for an efficient detection. (6-4)PPs are recognized relatively in ease because of having a prominent distortion \citep{mizukoshi2001structural}, whereas the distortion of CPDs cause only a 9\textsuperscript{o} unwinding with a 30\textsuperscript{o} bent \citep{park2002crystal}, which can be considered mild. For the detection of CPDs, proteins DDB1 and DDB2 form a complex called ultraviolet radiation–DNA damage-binding protein (UV-DDB). The complex directly interacts with the lesion, and DDB2 kinks the lesion to increase unwinding \citep{scrima2008structural}, as a result the ssDNA becomes detectable for XPC. 

The recognition mechanism of TCR is triggered by the blockage of RNA polymerase II (RNAPII), which transcribes the active gene during transcription elongation. When RNAPII stalls following an encounter with a lesion, it subsequently recruits the nucleotide excision repair proteins \citep{svejstrup2002mechanisms}. Afterwards, RNAPII dynamically interacts with UV-stimulated scaffold protein A (UVSSA), ubiquitin-specific-processing protease 7 (USP7), Cockayne syndrome protein CSB. CSB is an ATP-dependent chromatin remodeling factor that contains a helicase motif, surprisingly without a helicase activity \citep{selby1997human}. Moreover, studies in early 2000s revealed that point mutations in the ATPase domain of CSB protein significantly cripples the cell’s ability to escape the inhibited RNA synthesis \citep{citterio1998biochemical,muftuoglu2002phenotypic}, which suggests that CSB plays a key role for the TCR assembly. Furthermore, recruitment of repair factors that work on incision of the damaged fragment also mediated by CSB \citep{fousteri2006cockayne}. More identified functions of CSB include transcription elongation, chromatin maintenance and remodeling, histone tail binding, and strand annealing \citep{selby1997cockayne}. Another important Cockayne syndrome protein is CSA, which is also recruited by CSB. CSA mediates the recruitment of PCNA, RFC and pol $\delta$. Therefore, it is a key protein for the later events of the repair.

The recruited core nucleotide excision repair factors and some TCR specific factors such as UV-stimulated scaffold protein A (UVSSA), ubiquitin-specific-processing protease 7 (USP7), XPA-binding protein 2 (XAB2) and high mobility group nucleosome-binding domain-containing protein 1 (HMGN1), gather on the lesion site where RNAPIIo stalls. However, because RNAPIIo stalls on the lesion, it covers the lesion so that the TCR complex cannot reach it \citep{tornaletti1999structural}. To proceed, RNAPIIo should somehow move from the 35 nucleotides length of strand where it is positioned. There are three proposed mechanisms for that purpose which are degradation, dissociation and backtracking. Because backtracking is already known to be occurring at transcription proofreading and at natural transcription pause sites, it is the most accepted mechanism among these three \citep{marteijn2014understanding}. 

\subsubsection{Dual Incision and Excision of Damaged Fragment}

After RNAPIIo backtracks, transcription initiation factor IIH (TFIIH) initiates to unwind DNA with its helicase subunits. The TFIIH complex is formed of 10 proteins. While XPB and XPD have helicase activity, CDK-activating kinase (CAK) subcomplex is responsible for the initiation of TFIIH complex. The initiation is also known as DNA damage verification step which is the last reversible step of nucleotide excision repair \citep{marteijn2014understanding}. With the initiation of TFIIH complex, the lesion becomes ready to be removed. Then XPF-ERCC1 and XPC endonucleases interact with the lesion site to catalyze the lesion from two sides together with the TFIIH complex. Meanwhile, replication protein A (RPA) not only protects the non-damaged single strand, but also interacts with and coordinates most subunits of TFIIH complex. The cleavage of the lesion site that yields 22-30 nucleotide long single stranded gap, is termed dual incision \citep{marteijn2014understanding}.

\subsubsection{Re-synthesis and Ligation}

After the dual incision, the occurred gap must be filled with the ligation process. During replication, the proteins proliferating cell nuclear antigen (PCNA), replication factor C (RFC), DNA pol $\delta$, DNA pol $\epsilon$ and DNA ligase 1 mediates re-synthesis and ligation. However, if the cell is non-replicating, then DNA pol $\kappa$ and XRCC1– DNA ligase 3 fill the gap \citep{marteijn2014understanding}.

\subsection{NER associated diseases}

There are three human diseases that are known to be directly associated with nucleotide excision repair. These diseases are xeroderma pigmentosum (XP), cockayne syndrome (CS) and trichothiodystrophy (TTD) \citep{de2000nucleotide,lehmann2003dna}. 
XP discovered in 1968 as a hereditary disease that causes a defective nucleotide excision repair \citep{cleaver1968defective}. XP patients are extremely photosensitive, so that they have approximately 5000-fold increased risk of UV-induced skin cancer. Dry parchment skin and pigmentation related anomalies are some of the hallmarks of this disorder \citep{de2000nucleotide}.  Seven genes that are associated with the disease, known as XP complementation groups (XP-A, B, C, D, E, F, G) \citep{cleaver1975xeroderma}, and proteins that are produced by all these genes have a role in GR. Except XPC and XPE, they are also involved in TCR \citep{van1995transcription}.  

CS first reported in 1936 as a disease related to deafness and dwarfism \citep{cockayne1936dwarfism}. In the upcoming years, problems at joints, vision, and calcifications in the brain are further reported \citep{cockayne1946dwarfism,neill1950syndrome}. Moreover, these patients have aging related issues, and like XP patients, they are photosensitive, though not as severe as XP patients, therefore, their risk of having UV-induced skin cancer is not increased. As a consequence of all these abnormalities, most severe types of CS patients have a lifespan of as short as 7 years. Two genes, CSA and CSB are known to be related to the disease, which are both TCR proteins. Thereby, it was thought that CS patients are TCR defective. However, since TCR deficiency is not enough to explain all these severe symptoms alone, a deficiency in transcription is also argued \citep{drapkin1994dual}.

TTD patients can display a broad range of symptoms from having brittle hair to low fertility and impaired intelligence. If the disorder is caused by one of the XPB, XPD or TTDA genes, which are all code for a component of TFIIH complex, TTD patients can become nucleotide excision repair deficient, hence photosensitive. Even though TFIIH complex can be functional, the levels of TFIIH complex decreases significantly \citep{giglia2006dynamic}.

\section{Replication and its contribution to Mutagenesis}

Owing to many potential origins of replication (ORIs), a mammalian cell replicates in approximately 10 hours \citep{takebayashi2017anatomy}. During the cell division, only a portion of these ORIs fires, and they fire in an asynchronized manner except the ORIs that are in proximity to each other. By firing simultaneously, these close packed ORIs coordinates the replication of regions longer than mega bases, termed as “replication domains” \citep{jackson1998replicon}. Replication domains are divided into 4: early replication domains, late replication domains and the zones between these domains are up transition zones and down transition zones \citep{farkash2008global,hansen2010sequencing,hiratani2008global,koren2014genetic,nakayasu1989mapping,o1992dynamic}. Generally, the interior regions of the nucleus are replicated earlier than nuclear periphery regions, thus located at early replication domains \citep{dimitrova2002spatio}. Multiple studies indicated that these domains are differ each other in the mutation frequencies. Suggested by the genome-wide analysis of mutation rates, early replication domains have reduced levels of mutation comparing to late replication domains \citep{lawrence2013mutational,stamatoyannopoulos2009human}. Also, in most cancers, base substitution mutation elevates in late replication domains \citep{schuster2012chromatin}.

Replication is driven by replication forks which are formed when a predefined ORI fires \citep{langston2009whither}. Replication fork usually proceeds bidirectionally, with the coordinated work of polymerases $\epsilon$ and $\delta$. During the movement of the fork, polymerases $\epsilon$ continuously synthesizes leading strands, whereas polymerase $\delta$ discontinuously synthesizes lagging strands. Moreover, bidirectionality creates an asymmetric work labor, so that two polymerases work on opposite strands towards different directions. In other words, in the left replicating fork, polymerases $\epsilon$ proceeds on the plus strand, while in the right replicating fork, it progresses on the minus strand. Studies suggests that this asymmetric progress of polymerases around the associated ORI are reflected to the mutation profiles, where lagging strand reported to harbor more mutations than leading strand \citep{haradhvala2016mutational,lujan2012mismatch,reijns2015lagging,shinbrot2014exonuclease}. The occurred asymmetry on mutation profiles is reasoned by the error-prone bypass mechanism on the leading strand that makes it vulnerable to mutations \citep{seplyarskiy2019error}. Other studies argued that the attachment of helicase to leading strand increases the damage response, thus leading to effective repair of the strand \citep{hedglin2017eukaryotic,yeeles2013rescuing}. Furthermore, many mutational signatures are reported to have a significant replication strand asymmetry \citep{tomkova2018mutational}.

\section{Mapping Damage Formation and Nucleotide Excision Repair Events using Damage-seq and Excision-seq (XR-seq) Methods, Respectively}

Mapping of UV-induced damages and their repair is assential to understand the role of nucleotide excision repair on mutagenesis. Since the birth of the field of DNA repair, which began with the discovery of photolyase in 1958 \citep{rupert1958photoreactivation,sancar2016mechanisms}, many methods are introduced to map DNA damage and repair \citep{li2020methodologies}. However, not until the emergence of next-generation sequencing techniques, genome-wide mapping of DNA damage and repair at single-nucleotide resolution could be performed. Today, there are several methods that can accomplish the task. Among these methods, Damage-seq and Excision-seq (XR-seq) can map UV-induced DNA damages and repair of these damages by nucleotide excision repair, respectively, which will be explained in the subsections below.

\subsection{Damage-seq}

Damage-seq mechanism can sensitively detect a variety of DNA lesions such as CPDs, (6-4)PPs, and  cisplatins, mainly using the DNA polymerase II stalling to its advantage \citep{hu2016cisplatin}. In fact, the method can be adapted to any DNA damage that stalls DNA polymerase II, where the damage-specific antibody is present \citep{sancar2016mechanisms}. After the induction of the damage, the genomic DNA is sonicated, ligated to first primers, and denaturated. Then, damage sites are immunoprecipitated by damage-specific antibodies and enriched. Following the enrichment, a biotinylated primer is annealed and extended by a polymerase called Q5 DNA polymerase, which extends the primer until it reachs the damage without synthesizing the site of the damage. Next, a second adopter is ligated to the extended primer for amplification by PCR. Lastly, the amplified oligomers can be sequenced and analyzed.      

\subsection{Excision-seq (XR-seq)}

XR-seq method can measure the repair of DNA damages that is coordinated by nucleotide excision repair, using the 22-30 nt long exiced oligomers that are produced after the dual incision of lesion site \citep{hu2019genome,hu2016cisplatin}. Excised oligomers are immunoprecipitated by TFIIH and ligated by adaptors from both sides. Next, the oligomers are filtered according to the damage of interest by immunoprecipitating with damage-specific antibodies. Then, using photolyases, lesions of the left oligomers are reversed for a proper PCR amplification process and the oligomers are sequenced.