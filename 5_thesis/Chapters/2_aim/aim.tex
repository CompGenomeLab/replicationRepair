\setlength{\parindent}{0pt}
\chapter{\bf THE SCOPE OF THE THESIS}

Nucleotide excision repair is the sole mechanism for the removal of bulky adducts. In this study, to assess the influence of replication along with nucleotide excision repair on mutation distribution across replicated sites, we analyzed the \gls{dsseq} and \gls{xrseq} data in replicating cells within the context of replication timing. Damage and repair maps are generated for \gls{CPD}s and \gls{64}s from \gls{uv}-irradiated \gls{hela} cells synchronized at two stages of the cell cycle: early S phase, and late S phase. \gls{dsseq} locates and quantifies the regions of \gls{uv} induced \gls{CPD} and \gls{64} damages, while \gls{xrseq} captures excised oligomers of the damage site that are removed by the nucleotide excision repair. In combination, two methods provide the genome-wide distribution of \gls{uv}-induced damages and the differential repair frequency of these damaged sites. 

Initially, we examined the quality of the reads that are produced by \gls{dsseq} and \gls{xrseq} methods. After quality filtering and performing pre-analysis of damage and repair reads, we located the positions of damage and repair events throughout the human genome. Then, we used these positions together with datasets obtained from public sources to compare the repair rate of nucleotide excision repair in different regions.

In the first part of the study, we mapped the damage and repair events to the replication domains, where closely packed origins of replications fire in a synchronized manner, resulting in simultaneous replication of these \gls{mb}-sized regions. Then, we normalized repair events with corresponding damage quantities to eliminate the potential bias caused by the damage formation. By doing so, we managed to observe the differential repair rate between replication domains at different time points on a genome scale. We performed a similar analysis using chromatin states of \gls{hela} cells and examined how chromatin states effect the repair rate of replication domains, while moving from early to late S phase of cell cycle.

Secondly, we aimed to understand whether nucleotide exicision repair contribute to a replicative strand asymmetry. Because nucleotide excision repair is highly associated with melanoma cancers, replicative strand asymmetry of nucleotide excision repair can correlate with the mutation profiles of this tumor type. We retrieved a somatic melanoma mutation dataset, and quantified the mutations on approximately 20 \gls{kb}-sized initiation zones where origin of replications are closely positioned. We further separated these initiation zones into their corresponding replication domains before quantifying the mutations. This method enabled us both to compare the mutation count differences of replication domains, and to observe the mutational strand asymmetry on initiation zones. Next, we examined the strand asymmetry of damage and repair events separately on initiation zones. To assess if nucleotide composition of initiation zones contribute to the strand asymmetry, we simulated \gls{dsseq} and \gls{xrseq} reads, and compared the signal levels of these reads on initiation zones as well. Lastly, we calculated the repair rate by normalizing repair events with damage quantities to evaluate the asymmetry of relative repair, which we termed repair rate.