\setlength{\parindent}{0pt}
\chapter{\bf INTRODUCTION}

\section{UV-Induced Damages in Humans}

Ultraviolet (UV) light is a portion of electromagnetic (EM) spectrum which is emitted from the sun together with visible light and heat. Based on its wavelength, UV light divides into three subgroups: UVA (wavelength 315-400 nm), UVB (wavelength 280-315 nm), and UVC (wavelength 100-280 nm). While the less energetic UVA making up the majority of UV light passing the atmosphere, all UVC and approximately 90\% of UVB is either blocked or absorbed by the ozone layer. Even in these conditions, we are not fully protected from the damaging effects of UV light. So that, UV irradiation accounts for approximately 30.000 DNA lesions’ formation per cell per hour.

The most abundant UV lesions in cellular DNA are pyrimidine dimers \citep{kielbassa1997wavelength} that are formed by the covalent bonds between the adjacent pyrimidines \citep{whitmore2001effect}. There are two different types of pyrimidine dimers based on their chemical structure; one is called cyclobutene pyrimidine dimers (CPDs), and the other is called pyrimidine (6-4) pyrimidone photoproducts [(6-4)PPs]. While both UVC and UVB can induce these dimer formations, UVA is only capable to induce CPDs. Nonetheless, already formed (6-4)PPs can be converted into their Dewar valence isomers by the induction of UVA. Moreover, UVA can induce oxidative DNA damages through photosensitized reactions \citep{hu2017cartography}. Thanks to the developments of time resolved spectroscopy techniques in the recent years, dynamics of pyrimidine dimer formation is well known. The formation and biological properties of UV lesions will be briefly discussed in the subsections below.   

\subsection{Cyclobutene Pyrimidine Dimers (CPDs)} 

CPDs are the most frequent pyrimidine dimers that are arising from the covalent linkages between the consecutive pyrimidines, and it is characterized by the four-member ring structure double bonded from the pyrimidine 5 and 6 \citep{whitmore2001effect}. In vivo, CPDs can be observed in four different configurations: cis-syn, cis-anti, trans-syn, or trans-anti. \citep{khattak1972photochemical} While it is generally seen in cis-syn form when the DNA is double-stranded \citep{wacker1964organic}, in denatured DNA and single-stranded regions, trans-syn configuration exists \citep{taylor1988synthesis}. Although it is rare, nonadjacent pyrimidines can also form CPDs in single-stranded regions \citep{nguyen1988ultraviolet}. Moreover, different configurations can affect the ability of repair enzymes to recognize these lesions and correct them, which cause mutability differences between the configurations \citep{friedberg2005dna}. 

Apart from the configuration of the lesions, their dipyrimidine doublets are a key property of CPDs. There can be four different dipyrimidine doublets: TT, TC, CT, and CC which can be observed at different rates based the type of UV exposure or the nucleotide content of the DNA. According to the study of Douki and Cadet, double-stranded mammalian DNA produces TT, TC, CT, and CC CPDs in 100:50:25:10 ratios respectively, under UVC and UVB exposure \citep{douki2001individual}. In addition, under the UVA exposure, TT CPDs can account for approximately 90\% of CPDs \citep{mouret2010uva}. Although TT CPDs are the abundant products of UV exposure for mammalian DNA, the abundance might be greatly influenced by the GC percentage of the DNA. Conversely, in the bacterial DNA that possess a rich GC percentage, TT CPDs are not the major products of UV exposure \citep{patrick1977studies}.

\subsection{Pyrimidine (6-4) Pyrimidone Photoproducts [(6-4)PPs] and their Dewar Valence Isomers}

(6-4)PPs form with occurrence of a pyrimidone ring by the bonding between C6 position of the 5’-end base and C4 position of the 3’-end base. In fact, this structure forms indirectly of the UV exposure, after a cyclic reaction intermediate, which can be either an oxetane or azetidine, when 3’-end base is thymine or cytosine, respectively. Hence, (6-4)PPs appear thousand times slower than CPDs \citep{schreier2007thymine}. 

Under UVC and UVB exposure, formation of (6-4)PPs is approximately five time less than that of CPDs \citep{douki2001individual}. Moreover, TT dipyrimidines that are the most abundant sites for the CPDs, it is less frequent for (6-4)PPs. Instead, TC and CCs are the frequent sites for (6-4)PPs, while CT (6-4)PPs are uncommon. Another interesting property of (6-4)PPs is its conversion into Dewar valence isomers with the photoisomerization process \citep{taylor1987dna}. Although UVB irradiation can trigger the process, with the combination of UVB and UVA exposure, the yield increases significantly.

\section{Nucleotide Excision Repair in Humans}

Throughout the generations, cells manage to evolve highly specialized repair mechanisms to coup with variety of lesions that threatens the genome integrity and survival. Considering the diversity of these lesions, it would be unexpected to have only a single mechanism that can preserve the integrity of genome. Hence, there are several repair mechanisms that cells utilize which are eminently conserved between species. In the event of double strand breaks (DSBs), both strands will be removed, and the repair of such damages are demanding. There are two mechanisms that can triggered by DSBs: homologous recombination (HR), and non-homologous end-joining (NHEJ). HR mechanism uses the sister-chromatid as a template to repair DSBs in an error-free manner. If sister-chromatid is not available for use, NHEJ directs the fusion of broken ends in an error-prone manner. Although being error-prone, NHEJ is the dominant mechanism for DSB repair in mammals. One reason is the distant proximity of chromatids to each other, and the DNA folding that makes the homologous sequence less reachable. In addition, imperfect matches by HR can lead to tragic outcomes such as creating repeated sequences \citep{li2018mismatch}.

On the other hand, when a damage occurs at a single strand, the opposite strand can be used as a template. In such cases, DNA excision repair mechanisms remove the lesion site and re-synthesize the gap using the template strand. While oxidation, deamination and alkylation damages are repaired by Base Excision Repair (BER) \citep{klungland1999base}, mismatches that escape proofreading are identified and corrected by Mismatch Repair (MMR) \citep{modrich1997strand}, and bulky adducts caused by UV irradiation, environmental mutagens, and chemotherapeutic agents are removed by Nucleotide Excision Repair (NER) \citep{reardon2005nucleotide}. Nucleotide excision repair contains two sub pathways called Global Repair (GR) and Transcription-Coupled Repair (TCR) that differ from each other at the damage recognition step. TCR is specialized in recognizing adducts in transcribed regions, while GR can recognize bulky adducts at any site. All these mechanisms are essential for the repair of single-strand DNA breaks that account for the 75\% of daily occurred lesions \citep{tubbs2017endogenous}. Subsections below will address the assembly and main properties of NER in more detail.  

\subsection{Repair of UV-induced damages by Nucleotide Excision Repair}

Identified firstly at E. coli by two independent studies published in 1964 \citep{boyce1964release,setlow1964disappearance}, NER can repair variety of bulky adducts from UV-induced pyrimidine dimers to chemotherapeutic agents such as cisplatin \citep{yimit2019differential}. Although repair mechanisms are highly conserved among the species, NER in human appeared to be surprisingly different from that of E. coli. While E. coli contains three proteins (UvrA, B, C) for the incision of damaged fragment, human NER have sixteen proteins for the task. More interestingly, there is not an evolutionarily relevance between these human and E. coli proteins. Lastly, the excised fragment is usually around twelve nucleotide long in E. coli. For humans, the length is around 30 nucleotides \citep{sancar2016mechanisms}. Human NER mechanisms (GR and TCR) can be generally discussed in three steps: 1) damage recognition, 2) dual incision and excision of damaged fragment, and 3) re-synthesis and ligation.  

\subsubsection{Damage Recognition}

As it is mentioned earlier, GR and TCR have distinct damage recognition steps. GR mechanism scans the whole genome to detect helix distortions caused by bulky adducts, whereas TCR responses only to a stalled RNA polymerase II during transcription. 

In GR, three proteins (XPC, RAD23B, CETN2) work in coordination to recognize the lesion site \citep{sugasawa1998xeroderma}. XPC is the first protein to interact with the lesion by binding to the small single-stranded DNA (ssDNA) that is left unpaired caused by the pyrimidine dimer formation at the opposite strand. The ability of XPC to bind unpaired ssDNA enable the GR mechanism to detect variety of lesions since unpaired ssDNA is a common characteristic of bulky adducts. After XPC binding, RAD23B and CETN2 interact with and stabilize XPC. However, helix distortions must be apparent to XPC for an efficient detection. (6-4)PPs are recognized relatively in ease because of having a prominent distortion \citep{mizukoshi2001structural}, the distortion of CPDs cause only a 9\textsuperscript{o} unwinding with a 30\textsuperscript{o} bent \citep{park2002crystal}, which can be considered mild. For the detection of CPDs, proteins DDB1 and DDB2 forms a complex called ultraviolet radiation–DNA damage-binding protein (UV-DDB). The complex directly interacts with the lesion, and DDB2 kinks the lesion to increase unwinding \citep{scrima2008structural}, as a result the ssDNA becomes detectable for XPC. 

The recognition mechanism of TC-NER is triggered by the blockage of RNA polymerase II (RNAPII), which transcribes the active gene during transcription elongation. When RNAPII stalled following an encounter with a lesion, it subsequently recruits the NER proteins \citep{svejstrup2002mechanisms}. Afterwards, RNAPII dynamically interacts with UV-stimulated scaffold protein A (UVSSA), ubiquitin-specific-processing protease 7 (USP7), Cockayne syndrome protein CSB. CSB is an ATP-dependent chromatin remodeling factor that contains a helicase motif, surprisingly without a helicase activity \citep{selby1997human}. Studies in early 2000s revealed that point mutations in the ATPase domain of CSB protein significantly cripples the cell’s ability to escape the inhibited RNA synthesis \citep{citterio1998biochemical,muftuoglu2002phenotypic}. This founding suggests that CSB plays a key role for the TCR assembly. Furthermore, recruitment of repair factors that work on incision of the damaged fragment also mediated by CSB \citep{fousteri2006cockayne}. More identified functions of CSB include transcription elongation, chromatin maintenance and remodeling, histone tail binding, and strand annealing \citep{selby1997cockayne}. Another important Cockayne syndrome protein is CSA, which is also recruited by CSB. CSA mediates the recruitment of PCNA, RFC and pol $\delta$. Therefore, it is a key protein for the later events of the repair.

The recruited core NER factors and some TC-NER specific factors such as UV-stimulated scaffold protein A (UVSSA), ubiquitin-specific-processing protease 7 (USP7), XPA-binding protein 2 (XAB2) and high mobility group nucleosome-binding domain-containing protein 1 (HMGN1), gather on the lesion site where RNAPIIo stalls. However, because RNAPIIo stalls on the lesion, RNAPIIo covers the lesion so that the TC-NER complex cannot reach it \citep{tornaletti1999structural}. To proceed, RNAPIIo should somehow left the 35 nucleotides length of strand where it is positioned. There are three proposed mechanisms for that purpose which are degradation, dissociation and backtracking. Backtracking is the most accepted mechanism among these three because it is already known to be occurring at transcription proofreading and at natural transcription pause sites \citep{marteijn2014understanding}. 

\subsubsection{Dual Incision and Excision of Damaged Fragment}

After RNAPIIo bactracks, transcription initiation factor IIH (TFIIH) initiates to unwind DNA with its helicase subunits. The TFIIH complex is formed of 10 proteins. While XPB and XPD have helicase activity, CDK-activating kinase (CAK)  subcomplex is responsible for the iniatiation of TFIIH complex. The initiation is also known as DNA damage verification step which is the last reversible step of nucleotide excision repair [ref15]. With the initiation of TFIIH complex, the lesion becomes ready to be removed. Then XPF-ERCC1 and XPC endonucleases interacts with the lesion site to catalyze the lesion from two sides together with TFIIH complex. The cleavage of the lesion site that yields 22-30 nucleotide long single stranded gap, is termed dual incision. Throughout the process, replication protein A (RPA) not only protects the non-damaged single strand, but also interacts with and coordinates most subunits of TFIIH complex [ref15].

\subsubsection{Re-synthesis and Ligation}

After the dual incision, the occurred gap must be filled with the ligation process. During replication the replication proteins proliferating cell nuclear antigen (PCNA), replication factor C (RFC), DNA Pol $\delta$, DNA Pol $\epsilon$ and DNA ligase 1 mediates re-synthesis and ligation. However, if the cell is non-replicating, DNA Pol $\kappa$ and XRCC1– DNA ligase 3 fill the gap [ref15].

\subsection{NER associated diseases}

There are three human diseases that are known to be directly associated with NER mechanism. These diseases are Xeroderma pigmentosum (XP), Cockayne syndrome (CS) and trichothiodys trophy (TTD) \citep{de2000nucleotide,lehmann2003dna}. 
XP discovered in 1968 as a hereditary disease that causes a defective NER mechanism \citep{cleaver1968defective}. XP patients are extremely photosensitive, so that they have approximately 5000-fold increased risk of UV-induced skin cancer. Dry parchment skin and pigmentation related anomalies are some of the hallmarks of this disorder \citep{de2000nucleotide}.  There are seven genes that are associated with the disease, known as XP complementation groups (XP-A, B, C, D, E, F, G) \citep{cleaver1975xeroderma}. Proteins that are produced by all these genes have a role in GR, whereas proteins except XPC and XPE are also involved in TCR \citep{van1995transcription}.  

CS firstly reported in 1936 as a disease related to deafness and dwarfism \citep{cockayne1936dwarfism}. In the upcoming years, problems at joints, vision, and calcifications in the brain are further reported \citep{cockayne1946dwarfism,neill1950syndrome}. Moreover, these patients have aging related issues, and like XP patients, they are photosensitive, though not as severe as XP patients. In addition, their risk of having UV-induced skin cancer is not increased. As a consequence of all these abnormalities, Type 2 (most severe) CS patients’ lifespan is as short as 7 years. Two genes, CSA and CSB are known to be related to the disease, which are both TCR proteins. Thereby, it was thought that CS patients are TCR defective. However, TCR deficiency cannot explain all these severe symptoms by itself, and a deficiency in transcription is also suggested \citep{drapkin1994dual}.

TTD patients can display a broad range of symptoms from having brittle hair to low fertility and impaired intelligence. If the disorder caused by one of the XPB, XPD or TTDA genes, which are all code for a component of TFIIH complex, TTD patients become NER deficient, hence photosensitive. Even though TFIIH complex can be functional, the levels of TFIIH complex decreases significantly \citep{giglia2006dynamic}.

\subsection{Contribution of DNA Excision Repairs to Mutagenesis}

In deficiencies of both mismatch repair and nucleotide excision repair, there are specific mutational signatures associated which contribute to different cancer types \citep{helleday2014mechanisms}. Nucleotide excision repair associated signature 7 displays replication timing differences and replication related strand asymmetry. Furthermore, because early replication domains are more reachable relative to late replication domains, mismatch repair causes a mutation difference between these domains by effectively repairing the mismatches at early replication domains \citep{supek2015differential}. Similarly, TCR creates a transcriptional strand asymmetry by repairing adducts only at transcribed strands and leaving the opposite strand untouched \citep{zheng2014transcription}. Even though signature 7 is linked with DNA replication timing and strand asymmetry \citep{tomkova2018mutational}, the contribution of nucleotide excision repair to mutation differences during replication is still unclear.

\section{Replication and its contribution to Mutagenesis}

In about 8-10 hours, a mammalian DNA replicates itself during the S phase of the cell cycle \citep{takebayashi2017anatomy}, thanks to thousands of potential origins of replication (ORIs) that are distributed throughout the genome. Only a subset of these origins fire when cell divides. In this subset, ORIs do not fire at the same time as well. However, ORIs that are close to each other tend to fire in a synchronized manner, resulting in simultaneous replication of Mb-sized regions called "replication domains" \citep{jackson1998replicon}. The regions that are replicated during early S phase are called early replication domains, which are often located at the interior of the nucleus, while the nuclear periphery regions are replicated at late S phase, and therefore, they are called late replication domains \citep{dimitrova2002spatio,farkash2008global,hansen2010sequencing,hiratani2008global,koren2014genetic,nakayasu1989mapping,o1992dynamic}. Genome-wide analysis of the mutation rates suggested that mutation frequency increases at late replication domains \citep{lawrence2013mutational,stamatoyannopoulos2009human}. Moreover, late replication domains in a majority of cancers have elevated levels of base substitution mutations compared to early replication domains \citep{schuster2012chromatin}. 

In eukaryotes, replication initiates from predefined ORIs \citep{xiao2016iros,zhang2016iori} and once an ORI is fired, replication fork forms to coordinate DNA replication \citep{langston2009whither}. Replication fork proceeds in both directions, while polymerase epsilon continuously synthesizes leading strand, polymerase delta discontinuously synthesizes lagging strand until the fork collides to another that is fired from an adjacent ORI. As replication forks progress bidirectionally, polymerase epsilon and delta work on opposite strands towards different directions; while polymerase epsilon progresses on the plus strand in left-replicating fork, it moves on the minus strand in right-replicating fork, which creates an asymmetry on strands at the two sites around the associated ORI. Studies indicated that this asymmetric progress of the polymerases causes asymmetric mutation profiles as well  \citep{haradhvala2016mutational,lujan2012mismatch,reijns2015lagging,shinbrot2014exonuclease}. Furthermore, a significant replication strand asymmetry is shown to be prominent for the majority of the mutational signatures \citep{tomkova2018mutational}. As a reason of the asymmetric mutation profiles, a recent study suggested that lagging strand is prone to mutations, caused by the error-prone polymerase which bypasses the damaged site \citep{seplyarskiy2019error}. Moreover, another reason can be the helicase that attaches to leading strand during replication, and thus causing leading strand to be more responsive to damage \citep{hedglin2017eukaryotic,yeeles2013rescuing}. In general, all these effects caused by the strand asymmetric labor of polymerases around ORIs contribute to mutational footprints at those regions \citep{lujan2016dna}.

\section{Mapping Damage Formation and Nucleotide Excision Repair Events using Damage-seq and Excision-seq (XR-seq) Methods, Respectively}

Mapping of UV-induced damages and their repair is assential to understand the role of nucleotide excision repair on mutagenesis. Since the birth of the field of DNA repair, which began with the discovery of photolyase in 1958 \citep{rupert1958photoreactivation,sancar2016mechanisms}, many methods are introduced to map DNA damage and repair \citep{li2020methodologies}. However, not until the emergence of next-generation sequencing techniques, genome-wide mapping of DNA damage and repair at single-nucleotide resolution could be performed. Today, there are several methods that can accomplish the task. Among these methods, Damage-seq and Excision-seq (XR-seq) can map UV-induced DNA damages and repair of these damages by nucleotide excision repair, respectively, which will be explained in the subsections below.

\subsection{Damage-seq}

Damage-seq mechanism can sensitively detect variety of DNA lesions such as CPDs, (6-4)PPs, and  cisplatins, mainly using the DNA polymerase II stalling to its advantage \citep{hu2016cisplatin}. In fact, the method can be adapted to any DNA damage that stalls DNA polymerase II, where the damage-specific antibody is present \citep{sancar2016mechanisms}. After the induction of the damage, the genomic DNA is sonicated, ligated to first primers, and denaturated. Then, damage sites immunoprecipitated by damage-specific antibody and enriched. Following the enrichment, a biotinylated primer is annealed and extended by a polymerase called Q5 DNA polymerase, which extends the primer until it reachs the damage without synthesizing the site of the damage. Next, a second adopter is ligated to the extended primer for amplification by PCR. Lastly, the amplified oligomers can be sequenced and analyzed.      

\subsection{Excision-seq (XR-seq)}

XR-seq method can measure the repair of DNA damages that is coordinated by nucleotide excision repair, using the 22-30 nt long exiced oligomers that are produced after the dual incision of lesion site \citep{hu2019genome,hu2016cisplatin}. Excised oligomers are immunoprecipitated by TFIIH and ligated by adaptors from both sides. Next, the oligomers are filtered according to the damage of interest by immunoprecipitating with damage-specific antibody. Then, using photolyases, lesions of the left oligomers are reversed for a proper PCR amplification process and the oligomers are sequenced.