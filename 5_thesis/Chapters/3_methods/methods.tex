\setlength{\parindent}{0pt}
\chapter{\bf MATERIALS \& METHODS}

\section{Materials}


\subsection{Samples}
\begin{table}[]
    \begin{tabular}{cccccc}
    \hline
    \multicolumn{1}{|c|}{\textbf{cell line}} & \multicolumn{1}{c|}{\textbf{product}} & \multicolumn{1}{c|}{\textbf{method}} & \multicolumn{1}{c|}{\textbf{release}} & \multicolumn{1}{c|}{\textbf{time}} & \multicolumn{1}{c|}{\textbf{replicate}} \\ \hline
    HeLa-S3           & CPD              & XR-seq         & early            & 120           & A                  \\ \hline 
    HeLa-S3           & CPD              & XR-seq         & late             & 120           & A                  \\ \hline 
    HeLa-S3           & CPD              & XR-seq         & early            & 120           & B                  \\ \hline 
    HeLa-S3           & CPD              & XR-seq         & late             & 120           & B                  \\ \hline 
    HeLa-S3           & CPD              & Damage-seq     & early            & 120           & A                  \\ \hline 
    HeLa-S3           & CPD              & Damage-seq     & late             & 120           & A                  \\ \hline 
    HeLa-S3           & CPD              & Damage-seq     & early            & 120           & B                  \\ \hline 
    HeLa-S3           & CPD              & Damage-seq     & late             & 120           & B                  \\ \hline 
    HeLa-S3           & (6-4)PP           & XR-seq         & async            & 12            & A                  \\ \hline 
    HeLa-S3           & (6-4)PP           & XR-seq         & async            & 12            & B                  \\ \hline 
    HeLa-S3           & (6-4)PP           & XR-seq         & early            & 12            & A                  \\ \hline 
    HeLa-S3           & (6-4)PP           & XR-seq         & early            & 12            & B                  \\ \hline 
    HeLa-S3           & (6-4)PP           & XR-seq         & late             & 12            & A                  \\ \hline 
    HeLa-S3           & (6-4)PP           & XR-seq         & late             & 12            & B                  \\ \hline 
    HeLa-S3           & CPD              & XR-seq         & async            & 12            & A                  \\ \hline 
    HeLa-S3           & CPD              & XR-seq         & async            & 12            & B                  \\ \hline 
    HeLa-S3           & CPD              & XR-seq         & early            & 12            & A                  \\ \hline 
    HeLa-S3           & CPD              & XR-seq         & early            & 12            & B                  \\ \hline 
    HeLa-S3           & CPD              & XR-seq         & late             & 12            & A                  \\ \hline 
    HeLa-S3           & CPD              & XR-seq         & late             & 12            & B                  \\ \hline 
    HeLa-S3           & (6-4)PP           & Damage-seq     & async            & 12            & A                  \\ \hline 
    HeLa-S3           & (6-4)PP           & Damage-seq     & async            & 12            & B                  \\ \hline 
    HeLa-S3           & (6-4)PP           & Damage-seq     & early            & 12            & A                  \\ \hline 
    HeLa-S3           & (6-4)PP           & Damage-seq     & early            & 12            & B                  \\ \hline 
    HeLa-S3           & (6-4)PP           & Damage-seq     & late             & 12            & A                  \\ \hline 
    HeLa-S3           & (6-4)PP           & Damage-seq     & late             & 12            & B                  \\ \hline 
    HeLa-S3           & CPD              & Damage-seq     & async            & 12            & A                  \\ \hline 
    HeLa-S3           & CPD              & Damage-seq     & async            & 12            & B                  \\ \hline 
    HeLa-S3           & CPD              & Damage-seq     & early            & 12            & A                  \\ \hline 
    HeLa-S3           & CPD              & Damage-seq     & early            & 12            & B                  \\ \hline 
    HeLa-S3           & CPD              & Damage-seq     & late             & 12            & A                  \\ \hline 
    HeLa-S3           & CPD              & Damage-seq     & late             & 12            & B \\
    \hline                
    \end{tabular}
    \end{table}

\subsection{Programming Languages \& Tools}
\begin{table}[]
    \begin{tabular}{cccc}
    \hline
    \multicolumn{1}{|c|}{\textbf{\begin{tabular}[c]{@{}c@{}}Programming Languages \\ and Tools\end{tabular}}} & \multicolumn{1}{c|}{\textbf{Description}}                                                                                                 & \multicolumn{1}{c|}{\textbf{Purpose of use}}                                                                                                                                & \multicolumn{1}{c|}{\textbf{Source}} \\ \hline
    Bash                                                                                                      & a shell compatible command language                                                                                                       & \begin{tabular}[c]{@{}c@{}}constructing pipeline,\\ running tools,\\ format conversions\end{tabular}                                                                        & ref                                  \\
    Python                                                                                                    & a high-level, general purpose programming language                                                                                        & \begin{tabular}[c]{@{}c@{}}RPKM calculation,\\ aggregating windows of regions\end{tabular}                                                                                  & ref                                  \\
    R                                                                                                         & a language and an environment for graphics and statistics                                                                                 & \begin{tabular}[c]{@{}c@{}}plotting graphs,\\ correlation analysis\end{tabular}                                                                                             & ref                                  \\
    Cutadapt                                                                                                  & detects and cuts adaptor sequences                                                                                                        & \begin{tabular}[c]{@{}c@{}}removing adaptors,\\ discarding reads \\ containing adaptors\end{tabular}                                                                        & ref                                  \\
    Bowtie2                                                                                                   & a fast and memory-efficient sequence aligner                                                                                              & \begin{tabular}[c]{@{}c@{}}aligning reads to the \\ reference genome\end{tabular}                                                                                           & ref                                  \\
    Samtools                                                                                                  & \begin{tabular}[c]{@{}c@{}}a suit that contains utilities to interact with and \\ manipulate high-throughput sequencing data\end{tabular} & \begin{tabular}[c]{@{}c@{}}sorting,\\ filtering low \\ quality reads\end{tabular}                                                                                           & ref                                  \\
    Bedtools                                                                                                  & a set of utilities to perform genomic analysis                                                                                            & \begin{tabular}[c]{@{}c@{}}combining paired reads,\\ converting bed files to fasta format,\\ calculating genome coverage,\\ intersecting regions to each other\end{tabular} & ref                                  \\
    BedGraphToBigWig                                                                                          & converts bedGraph files to bigWig                                                                                                         & creating bigWig files for visualization                                                                                                                                     & ref                                  \\
    Art                                                                                                       & \begin{tabular}[c]{@{}c@{}}a simulation tool that creates a synthetic \\ high-throughput sequencing data\end{tabular}                     & simulating XR-seq and Damage-seq reads                                                                                                                                      & ref                                 
    \end{tabular}
    \end{table}

\subsection{Databases}
\begin{table}[]
    \begin{tabular}{ccc}
    \hline
    \multicolumn{1}{|c|}{\textbf{Databases}}                                                    & \multicolumn{1}{c|}{\textbf{Data Obtained}}                                                                                                          & \multicolumn{1}{c|}{\textbf{Source}} \\ \hline
    \begin{tabular}[c]{@{}c@{}}The European Bioinformatics \\ Institute FTP Server\end{tabular} & \begin{tabular}[c]{@{}c@{}}Genome Reference Consortium \\ Human Build 37 \\ (GRCh37)\end{tabular}                                                    & ref                                  \\
    \begin{tabular}[c]{@{}c@{}}Gene Expression Omnibus \\ (GEO)\end{tabular}                    & \begin{tabular}[c]{@{}c@{}}Repli-Seq data of HeLa-S3 \\ (accession no: GSE53984),\\ SNS-Seq data of HeLa-S3 \\ (accession no: GSE37757)\end{tabular} & ref                                  \\
    UCSC Genome Browser                                                                         & \begin{tabular}[c]{@{}c@{}}ChromHMM segmentation \\ from HeLa-S3 ChIP-Seq data\end{tabular}                                                          & ref                                  \\
    \begin{tabular}[c]{@{}c@{}}Sequence Read Archive \\ (SRA)\end{tabular}                      & \begin{tabular}[c]{@{}c@{}}OK-Seq data \\ (accession no: SRP065949)\end{tabular}                                                                     & ref                                  \\
    \begin{tabular}[c]{@{}c@{}}International Cancer \\ Genome Consortium \\ (ICGC)\end{tabular} & \begin{tabular}[c]{@{}c@{}}Simple somatic mutations \\ of Melanoma\end{tabular}                                                                      & ref                                 
    \end{tabular}
    \end{table}

\section{Methods}

The experiments were performed at SancarLab by our collaborators, whereas analyses of the data were carried by us. 

\subsection{Cell culture and treatments}
HeLa-S3 cell lines that were purchased from ATCC were cultured in DMEM medium supplemented with 10\% FBS and 1\% penicillin/streptomycin at 37\textsuperscript{o}C in a 5\% atmosphere CO\textsubscript{2} humidified chamber. By double-thymidine treatment, cells were synchronized at late G1 phase, and released into S phase after the removal of thymidine. Thymidine at 50\% confluence was added to the cells to a final concentration of 2 mM for the initial thymidine treatment. After 18 hours, the cells were washed with PBS for their release 18 hours after the initial thymidine treatment, and cultured in fresh medium for 9 hours. Then for 15 hours, cells were treated with 2mM thymidine and released into S phase for designated time before UV irradiation. Cells were irradiated with 20J/m\textsuperscript{2} of UVC, then collected either immediately or after incubation at 37\textsuperscript{o}C for designated time for the following assays.

\subsection{Flow cytometry analysis}
HeLa-S3 cell lines were initially trypsinized, and then PBS washed. After washing, for 2 hours, cells were fixed in 70\% (v/v) ethanol at -20\textsuperscript{o}C, then for 30 minutes, stained in the staining solution at room temperature. Lastly, the progression of the cells throughout the S phase was analyzed by a flow cytometer.

\subsection{Damage-seq and XR-seq libraries preparation and sequencing}
After HeLa-S3 cell lines were harvested in ice-cold PBS at designated time, Damage-seq and XR-seq methods were applied. For Damage-seq, using PureLink Genomic DNA Mini Kit, genomic DNA was taken out and then, cut into fragments by sonication using Q800 Sonicator. After sonication, DNA fragments (1$\mu$g) were subjected to end repair, dA-tailing and ligation using the first adaptor. Then, the fragments were denaturated and immunoprecipitated with either anti-(6-4)PP or anti-CPD antibody. A primer called Bio3U was bound to the fragment and extended with Q5 DNA polymerase until the primer reaches the lesion site. Next, the extended primer fragments were purified and annealed to oligo SH for subtractive hybridization process. After the substractive hybridization, oligo SH was removed using streptavidin C1 and the fragments were ligated to the second adapter for PCR amplification process. For XR-seq, cells were lysed with a homogenizer and centrifuged to remove chromatin DNA. To extract the nucleotide excision repair products, lysed cells were immunoprecipitated with anti-XPG antibody, which precipitates the excision products. Then, purified fragments were ligated with adaptors from both ends. The fragments were further immunoprecipitation with either anti-(6-4)PP or anti-CPD antibody and lesion sites were repaired by photolyase. After PCR amplification and gel purification, the products were sequenced via Hiseq 2000/2500 platform by the University of North Carolina High-Throughput Sequencing Facility, or Hiseq X platform by the WuXiNextCODE Company.  

\subsection{Damage-seq sequence pre-analysis}
The sequenced reads with adapter sequence GACTGGTTCCAATTGAAAGTGCTCTTCCGATCT at 5' end, were discarded via cutadapt with default parameters for both single-end and paired-end reads (Martin, 2011). The remaining reads were aligned to the hg19 human genome using bowtie2 with 4 threads (-p) (Langmead \& Salzberg, 2012). For paired-end reads, maximum fragment length (-X), which means the maximum accepted total length of mated reads and the gap between them, was chosen as 1000. Using samtools, aligned paired-end reads were converted to bam format, sorted using samtools sort -n command, and properly mapped reads with a mapping quality greater than 20 were filtered using the command samtools view -q 20 -bf 0x2 in the respective order (Li et al.). Then, resulting bam files were converted into bed format using bedtools bamtobed -bedpe -mate1 command (Quinlan \& Hall, 2010). The aligned single-end reads were directly converted into bam format after the removal of low quality reads (mapping quality smaller than 20) and further converted into bed format with bedtools bamtobed command (Quinlan \& Hall, 2010). Because the exact damage sites should be positioned at two nucleotides upstream of the reads (Li et al.), bedtools flank and slop command were used to obtain 10 nucleotide long positions bearing damage sites at the center (5. and 6. positions) (Quinlan \& Hall, 2010). The reads that have the same starting and ending positions, were reduced to a single read for deduplication and remaining reads were sorted with the command sort -u -k1,1 -k2,2n -k3,3n. Then, reads that did not contain dipyrimidines (TT, TC, CT, CC) at their damage site (5. and 6. positions) were filtered out to eliminate all the reads that do not harbor a UV damage. Lastly, only the reads that were aligned to common chromosomes (chromosome 1-22 + X) were held for further analysis.

\subsection{XR-seq sequence pre-analysis}
The adaptor sequence TGGAATTCTCGGGTGCCAAGGAACTCCAGTNNNNNNACGATCTCGTATGCCGTCTTCTGCTTG at the 3' of the reads were trimmed and sequences without the adaptor sequences were discarded using cutadapt with default parameters (Martin, 2011). Bowtie2 was used with 4 threads (-p) to align the reads to the hg19 human genome (Langmead \& Salzberg, 2012). Then reads with mapping quality smaller than 20 were removed by samtools (Li et al.). Bam files obtained from samtools were converted into bed format by bedtools (Quinlan \& Hall, 2010). Multiple reads that were aligned to the same position, were reduced to a single read to prevent duplication effect and remaining reads were sorted with the command sort -u -k1,1 -k2,2n -k3,3n. Lastly, only the reads that were aligned to common chromosomes were held for further analysis.

\subsection{Dna-seq sequence pre-analysis}
Paired-end reads were aligned to hg19 human genome via bowtie2 with 4 threads (-p) and maximum fragment length (-X) chosen as 1000 (Langmead \& Salzberg, 2012). Sam files were converted into bed format as it was performed at damage-seq paired-end reads. Duplicates were removed and reads were sorted with sort -u -k1,1 -k2,2n -k3,3n command. Lastly, the reads that did not align to the common chromosomes were discarded.

\subsection{XR-seq and Damage-seq simulation}
Art simulator was used to produce synthetic reads with the parameters
-l 26 -f 2, -l 10 -f 2 for XR-seq and Damage-seq, respectively (Huang, Li, Myers, \& Marth). To better represent our filtered real reads, read length (-l) parameter was chosen as the most frequent read length after pre-analysis done. The .fastq file that Art produced, were filtered according to our reads by calculating a score using nucleotide frequency of the real reads and obtaining most similar 10 million simulated reads. 
The filtering was done by filter\_syn\_fasta.go script, which is available at the repository: https://github.com/compGenomeLab/lemurRepair. 
Filtered files were preceded by pre-analysis again for further analysis.

\subsection{Quantification of melanoma mutations}
Melanoma somatic mutations of 183 tumor samples were obtained from the data portal of International Cancer Genome Consortium (ICGC) as compressed .tsv files which is publicly available at https://dcc.icgc.org/releases/release\_28/Projects/MELA-AU. Single base substitution mutations were extracted, and only the mutations of common chromosomes were used. To obtain the mutations that could be caused by UV-induced photoproducts, C -> T mutations that have a pyrimidine at the upstream nucleotide was further extracted. Later on, mutations were quantified on 20 kb long initiation zones that were separated into their corresponding replication domains using bedtools intersect command with the -wa -c -F 0.5 options.

\subsection{Further analysis}
In order to separate a region data (replication domains, initiation zones, or replication origins) into chosen number of (201) bins, the start and end positions of all the regions set to a desired range with the unix command: %awk -v a="$intervalLen" -v b="$windowNum" -v c="$name" '{print $1"\t"int(($2+$3)/2-a/2-a*(b-1)/2)"\t"int(($2+$3)/2+a/2+a*(b-1)/2)"\t"$4"\t"".""\t"$6}' 
Then, any intersecting regions or regions crossing the borders of its chromosomes were filtered to eliminate the possibility of signal's canceling out effect. After that, bedtools makewindows command was used with the -n 201 -i srcwinnum options to create a .bed file containing the bins.
To quantify the XR-seq and Damage-seq profiles on the prepared .bed file, bedtools intersect command was used to intersect as it was performed for mutation data. Then all bins were aggregated given their bin numbers, and the mean of the total value of each bin were calculated. Lastly RPKM normalization was performed and the plots were produced using ggplot2 in R programming language.