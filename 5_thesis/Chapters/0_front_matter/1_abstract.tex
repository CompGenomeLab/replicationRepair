\pagenumbering{roman}
\setcounter{page}{3}
\chapter*{\vspace{-4\baselineskip} \bf ABSTRACT} 
%\vspace{-1\baselineskip}

\begin{center}
\MakeUppercase{GENOME-WIDE EFFECTS OF DNA REPLICATION ON
NUCLEOTIDE EXCISION REPAIR OF UV-INDUCED
DNA LESIONS.} \\[3\baselineskip]
\MakeUppercase{Cem Azgari} \\[\baselineskip]
MOLECULAR BIOLOGY, GENETICS AND BIOENGINEERING M.S. THESIS, \Cemdateformat\today \\[\baselineskip]
Thesis Supervisor: Asst. Prof. Ogün Adebali \\[2\baselineskip]
Keywords: Nucleotide excision repair, UV damage, (6-4)PP, CPD, XR-seq, Damage-seq, DNA replication, DNA strand asymmetry \\[2\baselineskip]
\end{center}

\singlespacing
Replication can cause unrepaired DNA damages to turn into mutations that might lead to cancer. Nucleotide excision repair is the leading repair mechanism that prevents melanoma cancers by removing UV-induced bulky adducts. However, the role of replication on nucleotide excision repair, in general, is yet to be clarified. Recently developed methods Damage-seq and XR-seq map damage formation and nucleotide excision repair events respectively, in various conditions. Here, we applied Damage-seq and XR-seq methods to UV-irradiated HeLa cells synchronized at two stages of the cell cycle: early S phase, and late S phase. We analyzed the damage and repair events along with replication origins and replication domains of HeLa cells. We found out that in both early and late S phase cells, early replication domains are more efficiently repaired relative to late replication domains. The results also revealed that repair efficiency favors the leading strands around replication origins. Moreover, we observed that the repair efficiency of the strands around replication origins is inversely correlated with the number of melanoma mutations.

\clearpage\pagebreak
\onehalfspacing
