\chapter*{\vspace{-4\baselineskip} \bf ÖZET} 
%\vspace{-1\baselineskip}
\begin{otherlanguage}{turkish}
\begin{center}
\MakeUppercase{UV KAYNAKLI DNA HASARININ KESİP ÇIKARMALI ONARIMI İLE DNA REPLİKASYONUNUN GENOM ÇAPLI ETKİLEŞİMİ.} \\[3\baselineskip]
\MakeUppercase{CEM AZGARİ} \\[\baselineskip]
MOLEKÜLER BİYOLOJİ, GENETİK VE BİYOMÜHENDİSLİK YÜKSEK LİSANS TEZİ, \Cemdateformat\today \\[\baselineskip]
Tez Danışmanı: Dr. Ogün Adebali \\[2\baselineskip]
Anahtar Kelimeler: Nükleotid kesip çıkarmalı onarımı, UV hasarı, (6-4)PP, CPD, XR-seq, Damage-seq, DNA replikasyonu, DNA zinciri asimetrisi \\[2\baselineskip]
\end{center}

\singlespacing
Replikasyon, onarılmamış DNA hasarlarının kansere yol açabilecek mutasyonlara dönüşmesine neden olabilir. Nükleotid kesip çıkarmalı onarımı, UV ile hasara uğrayan hacimli DNA katımlarını ortadan kaldırarak melanom kanserlerini önleyen birincil onarım mekanizmasıdır. Ancak, replikasyonun nükleotid kesip çıkarmalı onarımındaki rolü henüz açığa kavuşturulmamıştır. Son zamanlarda geliştirilen Damage-seq ve XR-seq yöntemleri sırasıyla, hasar oluşumu ve nükleotid kesip çıkarmalı onarımını çeşitli koşullar altında haritalandırmıştır. Burada, Damage-seq ve XR-seq yöntemlerini hücre döngüsünün erken ve geç S fazlarında senkronize edilip UV ile hasara uğratılan HeLa hücrelerine uyguladık. HeLa hücrelerinin hasar ve onarım olaylarını replikasyon orijini ve replikasyon alanlarıyla birlikte analiz ettik. Hem erken hem de geç S fazlı hücrelerde, erken replikasyon alanlarının geç replikasyon alanlarına göre daha verimli bir şekilde onarıldığını ortaya çıkardık. Sonuçlar ayrıca onarım verimliliğinin replikasyon orijinleri etrafında DNA'nın öncü ipliklerini desteklediğini ortaya koydu. Dahası, replikasyon orijini etrafındaki ipliklerin onarım etkinliğinin melanom mutasyonlarının sayısı ile ters orantılı olduğunu gözlemledik. Özetle, bulgularımız, nükleotid kesip çıkarmalı onarımının, kanser genomunun replikasyonla ilişkili mutasyon zinciri asimetrisinde daha önce bilinmeyen bir rolü olduğunu ortaya koymaktadır.

\end{otherlanguage}

\clearpage\pagebreak
\onehalfspacing