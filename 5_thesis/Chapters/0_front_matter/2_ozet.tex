\chapter*{\vspace{-4\baselineskip} \bf ÖZET} 
%\vspace{-1\baselineskip}
\begin{otherlanguage}{turkish}
\begin{center}
\MakeUppercase{UV KAYNAKLI DNA HASARININ KESİP ÇIKARMALI ONARIMI İLE DNA REPLİKASYONUNUN GENOM ÇAPLI ETKİLEŞİMİ.} \\[3\baselineskip]
\MakeUppercase{CEM AZGARİ} \\[\baselineskip]
MOLEKÜLER BİYOLOJİ, GENETİK VE BİYOMÜHENDİSLİK YÜKSEK LİSANS TEZİ, \Cemdateformat\today \\[\baselineskip]
Tez Danışmanı: Dr. Ogün Adebali \\[2\baselineskip]
Anahtar Kelimeler: Nükleotid kesip çıkarmalı onarımı, UV hasarı, (6-4)PP, CPD, XR-seq, Damage-seq, DNA replikasyonu, DNA zinciri asimetrisi \\[2\baselineskip]
\end{center}

\singlespacing
Replikasyon, onarılmamış DNA hasarlarının kansere yol açabilecek mutasyonlara dönüşmesine neden olabilir. Nükleotid eksizyon onarımı, UV ile indüklenen hacimli DNA katımlarını ortadan kaldırarak melanom kanserlerini önleyen önde gelen onarım mekanizmasıdır. Ancak, replikasyonun Nükleotid kesip çıkarmalı onarımındaki rolü henüz açığa kavuşturulmamıştır. Son zamanlarda geliştirilen yöntemler Damage-seq ve XR-seq sırasıyla, hasar oluşumu ve nükleotid eksizyon onarımı olaylarını çeşitli koşullar altında haritalandırabilmektedir. Burada, Damage-seq ve XR-seq yöntemlerini hücre döngüsünün erken ve geç S fazlarında senkronize edilip UV ile indüklenen HeLa hücrelerine uyguladık. HeLa hücrelerinin hasar ve onarım olaylarını replikasyon orijini ve replikasyon alanlarıyla birlikte analiz ettik. Hem erken hem de geç S fazlı hücrelerde, erken replikasyon alanlarının geç replikasyon alanlarına göre daha verimli bir şekilde onarıldığını bulduk. Sonuçlar ayrıca onarım verimliliğinin replikasyon orijinleri etrafında DNA'nın öncü ipliklerini desteklediğini ortaya koydu. Dahası, replikasyon orijini etrafındaki ipliklerin onarım etkinliğinin melanom mutasyonlarının sayısı ile ters orantılı olduğunu gözlemledik.

\end{otherlanguage}

\clearpage\pagebreak
\onehalfspacing